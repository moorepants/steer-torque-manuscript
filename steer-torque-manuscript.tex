\documentclass[10pt]{article}

\usepackage{amsmath}
\usepackage{amssymb}
\usepackage{graphicx}
\usepackage{booktabs} % for table /toprule, /midrule, etc
\usepackage{pdflscape} % for the big table
\usepackage{siunitx}

% This file will not open in Acrobat Reader 9.x.
% This is an attempted fix:
% http://www.tug.org/pipermail/pdftex/2012-March/008806.html
\pdfobjcompresslevel=0

\begin{document}
\begin{flushleft}
  {\Large
  \textbf{Elimination of crosstalk and inertial effects in bicycle steer torque
    estimation}
  }
  \\
  Jason K. Moore$^{1,\ast}$,
  Mont Hubbard$^{2}$
  \\
  \bf{1} Mechanical Engineering, Cleveland State University, Cleveland, Ohio, USA
  \\
  \bf{2} Mechanical and Aerospace Engineering, University of California at Davis, Davis, California, USA
  \\
  $\ast$ E-mail: Corresponding moorepants@gmail.com
\end{flushleft}

\section*{Abstract}

% TODO : Write an abstract!

\section*{Introduction}

In bicycle and motorcycle control, the primary means of directing the vehicle
is application of force to the handlebars, causing steering rotation between
the front and rear frames. The rider can, of course, also use other more subtle
biomechanical movements to influence the vehicle trajectory, especially
lighter-weight ones, but it is well known that force applied directly to the
handlebars gives more control authority than other means
\cite{Weir1972,Aoki1979,Sharp2007,Sharp2008a}. From a single track vehicle
modeling perspective it is easiest to consider the human/machine interface
forces on the handlebar as an effectively lumped torque about the steer axis,
acting between the front and rear frames.

Due to this modeling assumption, most experimental measurements of the
interface forces between the rider and the front frame are aimed at estimating
the torque generated about the steering axis of the front frame. These
``direct'' measurements of rider applied steer torque are susceptible to two
major error sources: (1) inertial effects from the portion of the front frame
located between the sensor and the rider/vehicle interface, and (2) cross talk
from rider-applied forces other than those which generate steer torque.
Accounting for these error sources is particularly important when steer torques
are small ($<$~20 Nm or so) as is the case in most bicycle, and even some
motorcycle, maneuvers.

In this paper, we review previous methods of steer torque measurement in
bicycles and motorcycles and discuss the design and implementation of a bicycle
steer torque measurement system which minimizes the aforementioned cross talk
errors. We then provide the computations needed to correctly compensate for
inertial effects of the front frame and bearing friction to obtain a more
accurate estimate of the actual rider-applied steer torque. Finally, to show
the importance of this approach, we compare differences between uncompensated
and compensated steer torque measurements for a large set of bicycle
experiments.

\section*{History of Steer Torque Measurements}

% TODO: add et. al appropriately to the citations with multiple authors.

The earliest single track vehicle steer torque measurements were performed in
1951. Wilson-Jones~\cite{Wilson-Jones1951} developed a set of motorcycle
handlebars mounted in rubber bushings that indicated the direction and value of
torque in real time using an analog sensor much like a modern beam-style torque
wrench. He demonstrated that a torque applied in the opposite direction of the
eventual steer angle is used to enter a turn and measured torques in normal
maneuvers in the range of 4 to \SI{4}{\newton\meter}. Not long after, Kondo~\cite{Kondo1955}
was the first to record torque measurements electronically on a motorcycle for
post-experiment analysis. Work in Japan on motorcycle dynamics grew
considerably after World War II due to limitations on aircraft research and
Kageyama and Fu~\cite{Kageyama1959} and Fu~\cite{Fu1965} continued to improve
motorcycle steer torque measurements and studied these in steady turning.

% TODO: need citation on japan world war statement

Eaton~\cite{Eaton1973} was the first to measure and record motorcycle steer
torque in the US. He attached an additional handlebar above the stock
handlebars with strain gages that generated a voltage proportional to applied
torque around the steer axis, with the rider operating the motorcycle with one
hand. He measured steer torques up to 3.4 Nm for straight riding at speeds from
15 to 30 mph, leading him to conclude that most of the measured steer torque
was due to rider remnant, as opposed to deliberate control. Shortly afterward,
Weir, et al.~\cite{Weir1979a} developed a modular torque sensor that could be
used on multiple motorcycles, with a $\pm$ \SI{0}{\newton\meter} range, 1\% accuracy and 10 Hz
bandwidth. Weir was careful to reduce crosstalk from other forces applied to
the handlebars but the sensor was unfortunately oversized. Although the signal
to noise ratio was too low for steady turn and straight riding maneuvers,
torques of -20 to \SI{5}{\newton\meter} were measured in lane changes. Later Sugizaki and
Hasegawa~\cite{Sugizaki1988} measured steer torques in high speed motorcycle
lane change maneuvers between -20 and \SI{0}{\newton\meter}.

After nearly 50 years of steer torque measurements on motorcycles, the first
bicycle measurements were made by de Lorenzo~\cite{Lorenzo1997} on a downhill
mountain bicycle fitted with a custom strain-gauged handlebar that could
effectively measure torques about the bicycle's longitudinal and vertical axes.
His measurements show maximum steer and longitudinal handlebar torques of 7 and
\SI{5}{\newton\meter}, respectively, demonstrating that non-steer related forces applied to the
handlebars can be significantly higher that those used in steering.

Around the turn of the 21st century, Bortoluzzi, et al. ~\cite{Bortoluzzi2000}
designed a successful motorcycle steer torque transducer in which floating
handlebars engage the fork through a small strain-gaged cantilever beam, which
apparently was less susceptible to crosstalk than earlier designs. They found
torques as large as \SI{20}{\newton\meter} for slalom maneuvers at \SI{40}{\meter\per\second}.
About the same time, James~\cite{James2002} developed a secondary floating
handlebar with an integrated linear load cell to measure steer torques in an
off-road motorcycle to identify a model relating steer torque inputs to vehicle
dynamics.

In 2003, Cheng completed a comprehensive study on bicycle steer torque as an
undergraduate project~\cite{Cheng2003}. He began by simply attaching a torque
wrench to a bicycle front frame in left turns at speeds from 0 to 13 $m/s$, and
found most steering torques were less than \SI{5}{\newton\meter}. He then designed a floating
handlebar, very similar to that of James~\cite{James2002} , which engaged the
steer tube via a linear load cell configured to measure torques up to \SI{4}{\newton\meter}.
Cheng measured torques as large as \SI{1}{\newton\meter} and \SI{0}{\newton\meter} for steady turning at 4.5
$m/s$ and in sharp turns, repectively, confirming that bicycles require much
lower torques for maneuvering than motorcycles and also that his sensor was
considerably oversized.

Iuchi and Murakami~\cite{Iuchi2006} constructed a bicycle with a steer motor
that ``senses'' the rider's input for use in additive control. The
rider-applied steer torque was estimated from the motor torque and the
handlebar and motor moments of inertia ??? what does this mean???. Capitani, et
al.~\cite{Capitani2006} found measured steer torques on an instrumented scooter
between -15 and 40 $Nm$. Evertse~\cite{Evertse2010} was perhaps the only person
to estimate steer torque from sensors in the handle grips of a motorcycle that
measured forces directly at the human-vehicle interface. During his test
maneuvers, a maximum of \SI{0}{\newton\meter} was observed. In 2010, Teerhuis and
Jansen~\cite{Teerhuis2010}  measured torques just less than \SI{0}{\newton\meter} for a
motorcycle in slalom maneuvers using a floating handlebar plus linear load
tranducer design similar to those in \cite{James2002} and \cite{Cheng2003}.

Recently, Cain and Perkins~\cite{Cain2010,Cain2012} developed an
in-the-steer-tube bicycle torque sensor. Their measured steer torques in steady
turns never exceeded 2.4 Nm. van den Ouden~\cite{Ouden2011} also developed a
bicycle steer torque sensor susceptible to cross talk from other handlebar
loads but with a somewhat more appropriate measurement range of $\pm7.5$ Nm.
More recently, Appelman~\cite{Appelman2012} and Peterson~\cite{Peterson2013}
developed robotic bicycles in which steer torque could be estimated from DC
motor currents and properties.

% TODO: recheck van Ouden's measurement range and update here and in the table.

% TODO : Add Kageyama's design from his robot motorcycle (email him and ask
% what pub is best to cite)

Steering torque has thus been measured in relatively few bicycle experiments
and in not that many more motorcycles experiments. Few designs actually measure
the true rider-applied steer torque, as defined in most models. This is more
consequential for bicycles than motorcycles because the small torques used in
typical bicycle control are of the order of 5-10 Nm. van den
Ouden~\cite{Ouden2011}, in particular, showed how sensitive the torque
measurements are to other handlebar loads. Most measurement designs place the
sensor somewhere between the rider's hands and the ground contact point of the
wheel. Although this may seem a natural position at which to measure steer
torque, apparently no one except perhaps Iuchi and Murakami~\cite{Iuchi2006}
has accounted for the dynamic inertial effects of the front frame above the
sensor on the measurement. Evertse~\cite{Evertse2010} may have had the only
design which mitigates this inertial compensation issue completely due to the
proximity of the sensors to the rider's hands.


\section*{Classification of Steer Torque Designs}

The various designs mentioned in the previous section can be classified into
two major groups: (1) those in which the handlebars the rider interacts with
are essentially fixed to the front frame and (2) those in which the handlebars
float with respect to the front frame. Both of these groups can be further
categorized.

\subsection*{Fixed Handlebar Designs}

\begin{description}
  \item[Mechanical] The rider applies a force to the handlebar which causes a
    needle to indicate the torque in real
    time.~\cite{Wilson-Jones1951,Cheng2003,Moore2012}
  \item[Simple strain gages] Strain gages are applied to the vehicle's stock
    handlebars, typically on the steer column, to allow estimation of steer
    torque.~\cite{Eaton1973,Lorenzo1997,Capitani2006}
  \item[Load cells in the grips] Multiple degree of freedom load cells connect
    the handlebar grips to the handlebar.~\cite{Evertse2010}
  \item[Single load cell in steer column] A six degree of freedom load cell
    connects the handlebars to the front frame.~\cite{Kageyama?}
  \item[Estimated from motor power] An electrical motor connects the handlebars
    to the fork and the rider applied torque is estimated from motor current
    and voltage.~\cite{Iuchi2006,Appelman2012,Peterson2013}
\end{description}

% TODO: The mechanical type is sorta floating and sorta not. Wilson-Jones is
% fixed through rubber bushings, but a torque wrench may technically be
% floating on bearings.

\subsection*{Floating Handlebar Designs}

The floating designs all share the common feature that the handlebars are
connected to the front frame through a revolute joint aligned with steer axis
and all of the forces and torques are transmitted between the handlebars and
the front frame through a load cell of some sort.

\begin{description}
  \item[Inline torque transducer] The handlebars are separated from the fork,
    each is mounted in bearings, and the two are connected by an inline torque
    transducer.~\cite{Weir1979a,Cain2010,Cain2012,Moore2012}
  \item[Linear transducer] The handlebars are separated from the fork, each is
    mounted in bearings, and the two are connected through a linear force
    transducer.~\cite{Cheng2003,James2002,Teerhuis2010}
  \item[Cantilever transducer] The handlebars are separated from the fork, each
    is mounted in bearings, and the two are connected through a cantilever beam
    force transducer.~ \cite{Bortoluzzi2000,Biral2003,Ouden2011}
\end{description}

% TODO: add all citations to these types.

On examination of these previous efforts at steer torque measrement, we decided
to use the floating handlebar design concept with an inline torque transducer
but in such a way that no cross talk could affect the measurement and the
inertial effects of the structure between the rider's hands and the sensor were
properly accounted for. First, however, we did some exploratory experiments to
determine the appropriate range for the sensor, as there was not enough data in
the literature at the time for a confident choice.

\begin{landscape}
\begin{table}
  \caption{Summary of Past Steer Torque Measurments}
  \small
  \begin{tabular}{llllllll}
    \toprule
    Citation                              &  Year &     Vehicle &              Type & Sensor Range [Nm] & Max Measured Torque [Nm] &                               Maneuvers & Speed [m/s] \\
    \midrule
    \cite{Wilson-Jones1951}               &  1951 &  motorcycle &  fixed-mechanical &                NR &                       14 &  steady circles, entering/exiting turns &          NR \\
    \cite{Kondo1955}                      &  1955 &  motorcycle &                 ? &                 ? &                        ? &                   steady turns, turning &           ? \\
    \cite{Watanabe1973}                   &  1973 &  motorcycle &                 ? &                 ? &                       10 &                                 evasive &          14 \\
    \cite{Eaton1973}                      &  1973 &  motorcycle &      fixed-strain &                 ? &                      3.4 &     straight riding, roll stabilization &    6.7-13.4 \\
    \cite{Weir1979a}                      &  1979 &  motorcycle &      float-torque &         -70 to 70 &                30 and 55 &          steady turning and lane change &        > 10 \\
    \cite{Aoki1979}                       &  1979 &  motorcycle &                NR &                NR &                       98 &               straight, curving, slalom &       10-30 \\
    \cite{Sugizaki1988}                   &  1988 &  motorcycle &                 ? &                 ? &                       20 &                 high speed lane changes &       17-28 \\
    \cite{Lorenzo1997,Lorenzo1999}        &  1997 &     bicycle &      fixed-strain &       -500 to 500 &                        7 &                downhill mountain biking &         6-9 \\
    \cite{Bortoluzzi2000,Biral2003}       &  2000 &  motorcycle &  float-cantilever &                 ? &                       20 &                  steady turning, slalom &        6-40 \\
    \cite{James2002a,James2002,James2005} &  2002 &  motorcycle &      float-linear &         -30 to 30 &                       35 &                         straight riding &        2-19 \\
    \cite{Cheng2003}                      &  2003 &     bicycle &  fixed-mechanical &                   &         5 (20 on 13 m/s) &                   90 degree right turns &        0-13 \\
    \cite{Cheng2003}                      &  2003 &     bicycle &      float-linear &                84 &                 10 and 1 &          sharp turns and steady circles &         4.5 \\
    \cite{Iuchi2006}                      &  2006 &     bicycle &       fixed-motor &                NR &                       12 &                        straight running &             \\
    \cite{Capitani2006}                   &  2006 &     scooter &      fixed-strain &                NR &                       40 &                 lane change and turning &             \\
    \cite{Evertse2010}                    &  2010 &  motorcycle &       fixed-grips &                   &                       40 &                       obstacle manuever &             \\
    \cite{Teerhuis2010}                   &  2010 &  motorcycle &  float-cantilever &                NR &                       15 &                             steady turn &    8.3-12.5 \\
    \cite{Teerhuis2010}                   &  2010 &  motorcycle &  float-cantilever &                NR &                       20 &                                  slalom &   16.4-18.1 \\
    \cite{Ouden2011}                      &  2011 &     bicycle &  float-cantilever &                 5 &                        5 &               normal riding around town &             \\
    \cite{Moore2012}                      &  2012 &     bicycle &  fixed-mechanical &                   &                        5 &          steady circles, lane changes,  &             \\
    \cite{Cain2010,Cain2012}              &  2012 &     bicycle &      float-torque &    90\% oversized &                      2.4 &                            steady turns &             \\
    \cite{Moore2012}                      &  2012 &     bicycle &      float-torque &                   &   reported in this paper &           straight riding, lane changes &             \\
    \cite{Kageyama}                       &     ? &  motorcycle &         fixed-six &                   &                          &                                         &             \\
    \bottomrule
  \end{tabular}
  \label{tab:design}
\end{table}
\end{landscape}

\section*{Estimation of Bicycle Steer Torque Range}

Following Wilson-Jones~\cite{Wilson-Jones1951} and Cheng, et
al.~\cite{Cheng2003}, we used an accurate torque wrench as an auxiliary
handlebar to obtain a rough idea of the maximum torques during a variety of
maneuvers. We designed a simple attachment to the steer tube that allowed easy
connection of 3/8'' torque wrenches,
Figure~\ref{fig:torque-wrench-instrumentation}. A small video camera was
mounted to the bicycle so that it could record the dial indicator on the torque
wrench, the handlebars, and the digital speedometer from a viewpoint fixed in
the bicycle frame, Figure~\ref{fig:torque-wrench-instrumentation}.  The torque
wrench (CDI Torque Products 751LDIN) had a range of 1.7 Nm to 8.5 Nm and an
accuracy of $\pm2$\% of full scale ($\pm0.17$ Nm) for static measurements,
Figure~\ref{fig:torque-wrench-instrumentation}. The bicycle was a Surly 1 x 1
with speed maintained by an electric hub motor with cruise control so that no
pedaling was required.

\begin{figure}
  \centering
  \includegraphics[width=\textwidth]{figures/torque-wrench-instrumentation.png}
  \caption{{\bf Instrumentation used to measure steering torque.}
    The left image depicts the steer tube attachment in which the torque wrench
    is mounted. The middle image shows the torque wrench dial indicator,
    including the orange needle which stores the maximum torque value. The
    right image shows the camera mount and handlebars with torque wrench
    installed.
    }
  \label{fig:torque-wrench-instrumentation}
\end{figure}

We recorded video data for two riders performing seven different maneuvers:

\begin{itemize}
  \item tracking a straight line,
  \item transitions from straight running into tracking half circles of radii 6 and 10
    m,
  \item a 2 m lane change,
  \item a slalom with 3 m spacing between obstacles,
  \item and a steady tracking of circles with 5 and 10 m radius.
\end{itemize}

After the experiments we viewed the videos and manually noted the maximum and
minimum torques for each run, subjectively ignoring outlier high torque
readings from abnormal accelerations such as those due to riding over bumps. We
recorded run number, rider's name, rider's speed estimate, maximum value stored
on the torque wrench, maneuver type, minimum and maximum speeds from the video
footage , the maximum torque from the video footage, and the rotation sense
(clockwise [right turn] or counter clockwise). The raw data can be accessed via
Figshare~\cite{} and the Internet Archive~\cite{}.

The histogram, Figure \ref{fig:twrench-torque-histogram}, shows that the
maximum torque never exceeded \SI{5}{\newton\meter} in any run and Table
\ref{tab:maneuver-torque-values} gives the maximum and minimum torques for each
maneuver. Figure \ref{fig:twrench-torque-speed} shows all recorded torques as a
function of speed.

\begin{figure}
  \centering
  \includegraphics[width=4in]{figures/twrench-torque-histogram.pdf}
  \caption{{\bf Histogram of the maximum torque-wrench recorded torques for all runs.}
    The median is about \SI{2}{\newton\meter} with some torques measured as high as \SI{5}{\newton\meter}. }
  \label{fig:twrench-torque-histogram}
\end{figure}

\begin{table}
  \caption{Maximum and minimum torques values for different maneuvers.}
  \begin{tabular}{lrr}
    \toprule
    Maneuver & Maximum Torque [Nm] & Minimum Torque [Nm] \\
    \midrule
    Steady Circle (r = 10m) & 3.4 & -2.4 \\
    Steady Circle (r = 5m) & 2.4 & -2.2 \\
    Half Circle (r = 10m) & 3.8 & -5.0 \\
    Half Circle (r = 6m) & 3.4 & -5.0 \\
    Lane Change (2m) & 2.9 & -2.6 \\
    Line Tracking & 2.6 & -3.4 \\
    Slalom & 4.5 & -4.8 \\
    \bottomrule
  \end{tabular}
  \label{tab:maneuver-torque-values}
\end{table}

\begin{figure}
  \centering
  \includegraphics[width=6in]{figures/twrench-torque-speed.pdf}
  \caption{{\bf Maximum  and minimum torques generated by torque-wrench for
    each run as a function of speed.}
    }
  \label{fig:twrench-torque-speed}
\end{figure}

From this exploratory data, we concluded that a torque sensor with a range of
$\pm10$ Nm should give an adequate range with safety factor and still provide
high enough accuracy.

\section*{Isolated Steer Torque Measurement Design}

Our design incorporates a floating handle bar with inline torque sensor but
takes special care to isolate the torque measurement from non-contributing
loads. The torque transducer with a range of 150 in-lb ($\pm 17$ Nm) and
appropriate accuracy (TFF350, Futek, Irvine, CA, USA) ensures acceptable
accuracy for the low torques used in normal bicycle maneuvering. To guarantee
that torques are measured only about the steer axis, we isolated the steer
torque sensor from non-axial torques and all forces transmitted through the
handlebar or ground contact with a zero backlash telescoping double universal
joint, Figure~\ref{fig:steer-torque-design}. This universal joint  transmits
motion only along the main axis and has negligible backlash. ??? what IS
negligible???

\begin{figure}
  \centering
  \includegraphics[width=3in]{figures/steer-torque-design.png}
  \caption{{\bf The steer torque sensor isolation design.} The handlebars (not
    shown) attach to the stem mounted in the upper bearings. The fork and steer
    tube are mounted in the normal bicycle headset. Between these two sets of
    bearings the stem is connected to the steer tube via the torque sensor and
    a zero backlash telescoping double universal joint. Steer angle and rate
    are measured at the steer tube and the angular rate and acceleration of the
    rear frame are measured by the IMU.}
  \label{fig:steer-torque-design}
\end{figure}

\begin{figure}
  \centering
  \includegraphics[width=4in]{figures/steer-assembly.jpg}
  \caption{{\bf Steer torque measurement assembly.} }
  \label{fig:steer-torque-design}
\end{figure}

The bicycle fork is mounted as usual in the headset bearings, here called the
``lower''  bearings. A rigid frame structure was welded to the bicycle frame to
support the ``upper'' headset bearings. The handlebar's steer tube was mounted
in these tapered roller bearings directly in line with the steer axis. The
universal joint was attached to the fork's steer tube. The torque sensor then
connected the top of the universal joint with the bottom of the handlebar stem.
A slip clutch, ultimately disabled during all of the experiments due to
unacceptable backlash, was mounted between the pair of upper headset bearings
to provide possible sensor overload protection. This design effectively
isolated the torque sensor from all non-steer torque contributing loads.

Additional sensors were mounted on the rear frame and fork for kinematic
measurements needed for inertial compensation. The relative angle between the
bicycle front frame and rear frame, steer angle, was measured using a rotary
potentiometer. The body fixed angular rate of the front frame about the steer
axis was measured with a rate gyro. Body fixed 3D angular rate and 3D body
acceleration of a fixed point  on the rear frame was measured using an IMU.
Details of these sensors can be found in~\cite{Moore2012}.

% TODO : Report the accuracy of the sensor we chose.
% TODO : Add universal joint company
% TODO : Add a real picture of the final design to include beside the CAD
% drawing.
% TODO : Added slip clutch model and company.

\section*{Steering Dynamics}
\label{sec:steer-dynamics}

% TODO : Should the title of this section be ``Inertial Compensation''?

Because the final design placed the torque sensor inline with the steer axis
between the lower and upper head set bearings, the measured torque $T_M$ from
the sensor is not the same as the effective input torque applied by the rider.
Instead the rider applied steer torque $T_\delta$ can be shown to be a function
of the kinematics of both the front and rear frames and friction torques
generated by the bearings.

A free body diagram of the portion of the front frame assembly above the torque
sensor is shown in Figure~\ref{fig:handlebar-free-body}. Torques acting on the
handlebar about the steer axis are the measured torque $T_M$, the rider applied
steer torque $T_\delta$, and the friction from the upper bearing set $T_U$
which we approximate as a sum of Coulomb $T_{U_F}$ and viscous  $T_{U_V}$
friction.

\begin{figure}
  \centering
  \includegraphics{figures/handlebar-free-body.pdf}
  \caption{{\bf Free body diagram of the handlebar, $G$, with all axial torques
    shown.} The rear frame $B$ has arbitrary orientation with respect to the
    Newtonian reference frame $N$. The handlebar rotates about $\hat{b}_3$ with
    respect to the rear frame. Gravity $g$ is in the $\hat{n}_3$ direction. The
    rider applies forces to the handlebars, resulting in a component of torque
    $T_\delta$ about the steer axis. This torque is resisted by the upper
    bearing friction $T_U$, the measured torque at the sensor $T_M$, handlebar
    inertia, and the gravitational force acting at the handlebar center of mass
    $g_o$. The three distances, $d$, $d_{s1}$, and $d_{s3}$ locate the
    handlebar center of mass with respect to the center of the IMU at point
    $v$.}
  \label{fig:handlebar-free-body}
\end{figure}

We measure three body fixed components of angular velocity of the rear frame
$B$ in the Newtonian reference frame $N$ using three IMU rate gyros.  If
$\mathbf{b}_{1},\mathbf{b}_{2},\mathbf{b}_{3}$ are three orthonormal unit
vectors fixed in $B$, the angular velocity can be expressed as
%
\begin{equation}
  ^N\boldsymbol{\omega}^B = \omega_{b1}\hat{\mathbf{b}}_1 +
                            \omega_{b2}\hat{\mathbf{b}}_2 +
                            \omega_{b3}\hat{\mathbf{b}}_3
  \label{eq:rear-frame-angular-rate}
\end{equation}

The handlebar, frame $G$, is connected to the bicycle rear frame $B$ by a
revolute joint that rotates through the steering angle $\delta$ about the
$\hat{\mathbf{g}_3}$ axis. We measure both the steering angle with a
potentiometer and the component of the inertial angular rate of the handlebar
about the steer axis $\omega_{g3}$ with a rate gyro. The inertial angular
velocity of the handlebar can then be written as
%
\begin{equation}
  ^N\boldsymbol{\omega}^G =
    (\omega_{b1}c_\delta + \omega_{b2}s_\delta)\hat{\mathbf{g}}_1 +
    (-\omega_{b1}s_\delta + \omega_{b2}c_\delta)\hat{\mathbf{g}}_2 +
    \omega_{g3}\hat{\mathbf{g}}_3
\end{equation}
%
where $c_\delta$ and $s_\delta$ are abbreviations for
$\operatorname{cos}(\delta)$ and $\operatorname{sin}(\delta)$ respectively.

The steer rate, $\dot{\delta}$, can be computed from the measurements by
subtracting the components of the inertial angular velocities of the bicycle
rear frame and the handlebar, both about the steer axis.
%
\begin{equation}
  \dot{\delta} \hat{\mathbf{b}}_3 = \omega_{g3} \hat{\mathbf{b}}_3 - \omega_{b3} \hat{\mathbf{b}}_3
\end{equation}

Now we define point $s$ on the steer axis at a minimum distance $d$ from the
center of mass $g_o$ of the handlebar. The vector from $s$ to $g_o$ is given by 
%
\begin{equation}
  \mathbf{r}^{g_o/s} = d\hat{\mathbf{g}}_1
\end{equation}

We also measure the body fixed acceleration of a point $v$ in $N$ on the
bicycle frame with the IMU accelerometers which includes an acceleration due to
gravity. This acceleration can be expressed as
%
\begin{equation}
  ^N\mathbf{a}^v =
    a_{v1}\hat{\mathbf{b}}_1 +
    a_{v2}\hat{\mathbf{b}}_2 +
    a_{v3}\hat{\mathbf{b}}_3
  \label{eq:acceleration-of-v}
\end{equation}

The location of point $s$ is known with respect to $v$ 
%
\begin{equation}
  \mathbf{r}^{s/v} = d_{s1}\hat{\mathbf{b}}_1 + d_{s3}\hat{\mathbf{b}}_3
\end{equation}

The acceleration of the handlebar center of mass, $^N\mathbf{a}^{g_o}$, can now
be calculated using the two point theorem for acceleration \cite{Kane1985}
%
\begin{equation}
  ^N\mathbf{a}^{g_o} = {}^N\mathbf{a}^s +
    {}^N\dot{\boldsymbol{\omega}}^G\times\mathbf{r}^{g_o/s} +
    {}^N\boldsymbol{\omega}^G\times({}^N\boldsymbol{\omega}^G\times\mathbf{r}^{g_o/s})
\end{equation}
where
\begin{equation}
  ^N\mathbf{a}^s = {}^N\mathbf{a}^v +
    {}^N\dot{\boldsymbol{\omega}}^B\times\mathbf{r}^{s/v} +
    {}^N\boldsymbol{\omega}^B\times({}^N\boldsymbol{\omega}^B\times\mathbf{r}^{s/v})
\end{equation}

The angular momentum of the handlebar about its center of mass is defined as
%
\begin{equation}
  ^N\mathbf{H}^{G/g_o} = I^{G/g_o} \cdot {}^N\boldsymbol{\omega}^G
\end{equation}
%
where $I^{G/g_o}$ is the inertia dyadic of the handlebar with respect to its
center of mass. We assume that the handlebar exhibits symmetry about the plane
normal to $\hat{\mathbf{g}}_2$, simplifying the inertia dyadic to
%
\begin{equation}
  I^{G/g_o} =
    I_{G_{11}} \hat{\mathbf{g}}_1 \otimes \hat{\mathbf{g}}_1 +
    I_{G_{22}} \hat{\mathbf{g}}_2 \otimes \hat{\mathbf{g}}_2 +
    I_{G_{33}} \hat{\mathbf{g}}_3 \otimes \hat{\mathbf{g}}_3 +
    I_{G_{31}} \hat{\mathbf{g}}_1 \otimes \hat{\mathbf{g}}_3 +
    I_{G_{31}} \hat{\mathbf{g}}_3 \otimes \hat{\mathbf{g}}_1
\end{equation}

Now the rotational dynamic equations of motion of the handlebar can be written
as the sum of the torques on the handlebar about point $s$ equals the
derivative of the angular momentum of $G$ in $N$ about $g_o$ plus the cross
product of the position vector from $s$ to $g_o$ with the mass times the
acceleration of $g_o$ in $N$ \cite{Meriam1975}. We neglect the torque due to
the weight at $g_o$ because it is accounted for in the accelerometer
measurement, Equation~\ref{eq:acceleration-of-v}.
%
\begin{equation}
  \sum \mathbf{T}^{G/s} = {}^N\dot{\mathbf{H}}^{G/g_o} +
    \mathbf{r}^{g_o/s} \times m_G\,{}^N\mathbf{a}^{g_o}
\end{equation}

We are only interested in the components of the previous equation in which the
steer torque appears, so only the torque components about the steer axis are
examined.
%
\begin{equation}
  \sum T^{G/s}_3 = T_\delta - T_U - T_M = \left({}^N\dot{\mathbf{H}}^{G/g_o} +
  \mathbf{r}^{g_o/s} \times m_G\,{}^N\mathbf{a}^{g_o}\right) \cdot \hat{\mathbf{g}}_3
\end{equation}

At this point the rider's applied steer torque, $T_\delta$, can be solved for,
giving
%
\begin{align}
  T_{\delta} =
    & I_{G_{22}} \left[ \left( -\omega_{b1} s_\delta + \omega_{b2} c_\delta \right)
      c_\delta + \omega_{b2} s_\delta \right] + I_{G_{33}} \dot{\omega}_{g3} + \nonumber \\
    & I_{G_{31}} \left[ (-\omega_{g3} + \omega_{b3} ) \omega_{b1} s_\delta +
      (-\omega_{b3} + \omega_{g3}) \omega_{b2} c_\delta +
      s_\delta \dot{\omega}_{b2} + c_\delta \dot{\omega}_{b1} \right] + \nonumber \\
    & \left[ I_{G_{11}} (\omega_{b1} c_\delta + \omega_{b2}s_\delta) +
      I_{G_{31}} \omega_{g3} \right] \left[-\omega_{b1} s_\delta +
      \omega_{b2} c_\delta \right] + \nonumber \\
    & d m_G \left[ d (-\omega_{b1} s_\delta + \omega_{b2} c_\delta)
      (\omega_{b1} c_\delta + \omega_{b2} s_\delta) + d \dot{\omega}_{g3} \right] - \nonumber \\
    & d m_G \left[-d_{s1} \omega_{b2}^{2} + d_{s3} \dot{\omega}_{b2} -
      (d_{s1} \omega_{b3} - d_{s3} \omega_{b1}) \omega_{b3} + a_{v1} \right] s_\delta + \nonumber \\
    & d m_G \left[d_{s1} \omega_{b1} \omega_{b2} + d_{s1} \dot{\omega}_{b3} +
      d_{s3} \omega_{b2} \omega_{b3} - d_{s3} \dot{\omega}_{b1} + a_{v2} \right]
      c_\delta + \nonumber \\
    & T_U + T_M
    \label{eq:steer-torque}
\end{align}

All parameters which are constant with respect to time in can be found
independently, see Table~\ref{tab:numerical-constants}. These distances,
masses, and inertias are determined as described in \cite{Moore2012}. All time
varying terms in Equation~\ref{eq:steer-torque} are either measured directly by
the on-board sensors or can be calculated by numerical differentiation. The
only exception is the upper bearing frictional torque, $T_U$. We estimate this
torque contribution with experiments described in the following section.

\begin{table}
  \centering
  \caption{Numerical values for the constant parameters in Equation
    \ref{eq:steer-torque} for the instrumented bicycle used in the experiments.
    These parameters were obtained using measurements explained in
    \cite{Moore2012}.}
  \begin{tabular}{lll}
    \toprule
    Variable     & Value  & Unit \\
    \midrule
    $d$          & 0.963  & m\\
    $d_{s1}$     & 0.135  & m \\
    $d_{s3}$     & -0.360 & m \\
    $m_G$        & 3.35   & kg \\
    $I_{G_{11}}$ & 0.0497 & $kg \cdot m^2$ \\
    $I_{G_{13}}$ & 0.0049 & $kg \cdot m^2$ \\
    $I_{G_{22}}$ & 0.0174 & $kg \cdot m^2$ \\
    $I_{G_{33}}$ & 0.0622 & $kg \cdot m^2$ \\
    \bottomrule
  \end{tabular}
  \label{tab:numerical-constants}
\end{table}

\section*{Estimation of Bearing Friction}
\label{sec:bearing-friction}

In our design, the torque sensor is mounted between two sets of bearings. The
upper set for the handlebars are tapered roller bearings and the lower are
typical bicycle headset bearings. Each is preloaded a nominal amount during
installation. We assume that the rotational friction torque due to each bearing
set can be described as the sum of viscous $T_{Bv}$ and Coulomb friction
$T_{Bc}$. The Coulomb friction is approximated as a piecewise-constant function
of steering rate, Equation~\ref{eq:coulomb}, and viscous friction as a linear
function of steer rate, Equation~\ref{eq:viscous}.

\begin{equation}
  T_{Bc} = t_B \operatorname{sgn}(\dot\delta) =
  \begin{cases}
    t_B  & \textrm{if $\dot{\delta}>0$}\\
    0    & \textrm{if $\dot{\delta}=0$}\\
    -t_B & \textrm{if $\dot{\delta}<0$}
  \end{cases}
  \label{eq:coulomb}
\end{equation}

\begin{equation}
  \label{eq:viscous}
  T_{Bv} = c_B \dot{\delta}
\end{equation}

The total friction (due to all four headset bearings) is

\begin{equation}
  \label{eq:total-friction}
  T_B = T_{Bc} + T_{Bv}
\end{equation}

To estimate constants $t_B$ and $c_B$, we mounted the bicycle with the steer
axis vertical, the front wheel off the ground, and the rear frame rigidly fixed
in inertial space. We then attached two parallel springs of stiffness $k$ to
the left handlebar so that the restoring force from the springs acted through
lever arm $l$ relative to the steer axis.

% TODO : Add a diagram or image of the experimental setup.

This configuration allowed application of small angular perturbations to the
handlebars and subsequent measurement of damped vibrations in steer angle
$\delta$, steer rate $\dot{\delta}$, and steer tube torque $T_M$. The equations
of motion governing this vibratory system are
%
\begin{equation}
  I_{HF} \ddot{\delta} + c_B \dot{\delta} + t_B
  \operatorname{sgn}(\dot{\delta}) + 2 k l^2 \delta = 0
\end{equation}
%
where $I_{HF}$ is the total moment of inertia of the combined front frame and
front wheel about the steer axis.

We measured the lever arm and spring stiffness as 0.213 meters and $904.7 \pm
0.6$ N/m respectively. The inertia of the handlebar, fork, and front wheel
about the steer axis, $I_{HF}$, was estimated based on measurements described
in \cite{Moore2012} and found to be $0.1297 \pm 0.0005$ $kg\cdot m^2$

We estimated the friction coefficients using a non-linear grey box
identification ????reference???? based on the measured steer angle vibrations
over 15 trials in which the steering assembly was perturbed from equilibrium.
The identified viscous coefficient is $c_B = 0.34 \pm 0.04$ $N \cdot m \cdot
s^2$ and the Coulomb coefficient is $t_B = 0.15 \pm 0.05$ $N \cdot m$.

To calculate the applied steer torque $T_\delta$ we need an estimate of the
upper bearing friction $T_U$ and at this point we only have the total friction
from both the lower and upper bearings. We made the simple assumption that the
friction in the upper and lower bearings are equal, $T_U = T_B / 2$, due to
indeterminacy of the upper and lower bearing friction individually from the
steer tube vibration measurements; see \cite{Moore2012} for details. It is
important to note that the upper bearings are preloaded, but that no other
signficant axial loads are applied in normal riding, thus our estimate of the
upper bearing friction is not dependent on having the wheel contact loaded
during these experiments.

% TODO : Maybe I should show a plot of one of the fits or something.

\section*{Steer Torque Predictions}

Using equations \ref{eq:steer-torque}-\ref{eq:total-friction} described in the
previous section \ref{sec:steer-dynamics} and estimates for the upper bearing
friction coefficients in \ref{sec:bearing-friction} ???? which ??? we compute
the compensated steer torque for 359 trials using data collected from the
instrumented bicycle set presented in \cite{Moore2012}.

Figure~\ref{fig:steer-torque-components} shows results from an example trial.
???? need more discussion of this fugure ????  We then compute, for each trial,
the root mean square of the error between the torque from the sensor and the
compensated torque. We also compute the maximum of the absolute value of the
error for each trial and the coefficient of determination (i.e. $R^2$) between
the compensated and uncompensated torques. Outliers beyond $\pm2 \sigma$ were
excluded from the results. ???? meaning ???? Figure \ref{fig:error-stats} shows
the distribution of these statistics. The median values of the three statistics
are given in Table~\ref{tab:medians}.

\begin{figure}
  \centering
  \includegraphics{figures/steer-torque-components.pdf}
  \caption{{\bf Steer torque measurements and the computed compensation for Trial \#
    700.  ????This figure caption needs more explanation????}}
  \label{fig:steer-torque-components}
\end{figure}

\begin{figure}
  \centering
  \includegraphics{figures/error-stats.pdf}
  \caption{{\bf Histograms of the three statistics for all 359 trials.}}
  \label{fig:error-stats}
\end{figure}

\begin{table}
  \caption{{\bf The median and maximum value of the error statistics.}}
  \centering
  \begin{tabular}{lrr}
    \hline
    Statistic                    & Median   & Maximum \\
    \hline
    Coefficient of Determination & 0.728 & 0.822 \\
    Maximum Error                & 2.446 & 6.588 \\
    RMS of the Errors            & 0.466 & 0.899
  \end{tabular}
  \label{tab:medians}
\end{table}

\section*{Discussion}

For the bicycle and maneuvers performed in the experiments described herein we
have shown that neglecting to compensate for inertial effects can have a large
influence on the accuracy of the results. In particular, on median ???
meaning??? 28\% of the actual torque applied by the rider in our experiments
would be neglected. This may be less important for motorcycles because the
nominal steer torques are usually much larger, but this error will always be
significant for measurements of low torque ($<$ \SI{0}{\newton\meter} or so) in any vehicle.
Steer torque sensor designs should account for the inertial effects of the
handlebars and eliminate crosstalk for high accuracy. Ideally, one would
measure the forces at the hand/handlebar interface with very accurate
six-component load cells and inertial compensation would not be necessary. The
closer the sensor is to the rider's hands the less important inertial
compensation becomes. But if more traditional direct steer torque measurements
are used, both inertial compensation and cross talk mitigation (mechanically or
computationally) will be needed. We have found only a couple of designs that
mitigate the inertial issue before us, namely \cite{Evertse2010} and
\cite{Iuchi2006}, and many previous design were aware of crosstalk, but no
design shows complete elimination as we have. Maneuvers with high steer
accelerations and high handlebar axial moments of inertia are especially
susceptible, due to the dominance of the effects of angular acceleration on
torque. This is clearly shown in Figure~\ref{fig:steer-torque-components}. Our
design gave very accurate torque measurements, but there is still much room for
improvement, especially in terms of complexity and cost. Creating a simple,
inexpensive, and accurate steer torque measurement system could play an
important role in assistive control system design in production vehicles.

\section*{Acknowledgements}

This paper is based on work supported by the National Science Foundation under
Grant No. 0928339. Thanks to Peter de Lange for developing the bearing friction
experimental protocol and analysis, Gilbert Gede and Mohammed Osman for
assistance in constructions and design, and to Ton van den Bogert for comments
and review.

\bibliographystyle{unsrt}
\bibliography{references}

\end{document}
